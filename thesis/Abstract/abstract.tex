% ************************** Thesis Abstract *****************************
% Use `abstract' as an option in the document class to print only the titlepage and the abstract.
\begin{abstract}
During the last fifty years, the scientific community has dealt with the problem of designing \textbf{humanoid robots}, with the goal of creating artificial machines flexible enough to master any task performed by a human. A prerequisite for such \emph{generic} machines would be the ability to control the forces that robots exchange with the environment, in addition to their own motion. These requirements imply that any \emph{controller} of a humanoid robot requires an \emph{implicit} or \emph{explicit} \textbf{model} for the robot's \textbf{dynamics}, i.e. the laws describing the relation between the robot motion and the forces applied on itself, being either the \emph{external} forces that the robot exchange with the environment or the forces provided by its own motors. Additionally, any time-variant quantity present in these models need to be \emph{perceived} by the humanoid robot. If a quantity is not directly measured by a sensor, it needs to be \textbf{estimated} using the available measurements of different quantities and the dynamical models that relate them.  Both the dynamics models and the sensor models are typically not perfectly known, and need to be \textbf{identified} from measured data and using a set of a-priori hypothesis. This thesis focuses on addressing the problems of modelling, estimation and identification of humanoid robots, focusing in particular to the specific characteristics and sensor set of the iCub humanoid robot. 
\end{abstract}
