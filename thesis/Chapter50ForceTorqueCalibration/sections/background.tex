\section{Problem statement and assumptions}
\label{sec:ft-calib-background}

We assume that the model for predicting the force-torque (also called wrench)  
% applied to the sensor 
from the raw measurements is an affine model, i.e. 
% \begin{equation}
% \rawval = \shapemat \hspace{0.2em} \senswrench -  {o_r}
% \end{equation} 
% Consequently, the output of the sensor can be computed as $\nrofsg$ 
%(with $m \geq 6$) 
% raw strain gauges values is :
\begin{equation}
\label{wrenchInSensorCoordinates}
\rmf =  C ( r - \offset ),
% =  {C} \left( \rawval -  {o_r} \right)
\end{equation} 
where
% \begin{itemize}
% \item 
$\rmf \in \mathbb{R}^{6}$ is the wrench exerted on the sensor expressed in the sensor's frame,
% \item 
$r \in \mathbb{R}^{\nrofsg}$ is the raw output of the sensor,
% \item 
$ {C} \in \mathbb{R}^{6 \times \nrofsg}$ is the
%full row rank 
invertible
calibration matrix, and
% \item 
$\offset \in \mathbb{R}^6$ is the sensor's bias or offset.
The calibration matrix and the offset are assumed to be constant.
% \end{itemize}

We assume that the sensor is attached to a rigid body
% mechanical structure --~such as a robot manipulator, 
% see Figure~\ref{photoOfTheIcubLegOrArmWhereYouSeeTheMachanincalChainAttachedToIt}~-- and that
% this structure posses $n$ degrees of freedoms. The associated configuration of this structure can be
% then characterized by an $n$-dimensional column vector $q \in \mathbb{R}^n$. 
% The mechanical structure has a total 
of 
(constant) mass $m~\in~\mathbb{R}^+$ and with a center of mass whose position
w.r.t. the sensor frame ${S}$ is characterized the vector $c \in \mathbb{R}^3$.

% \[{\bf{OC}}(q) = ({\bf{i_0}},{\bf{j_0}},{\bf{k_0}})\bar{c}(q) = ({\bf{i}},{\bf{j}},{\bf{k}})c(q).\]
% \[\bar{c} = Tc,\]
% with $\bar{c},c \in \mathbb{R}^3$ the vector of the center of mass expressed w.r.t. the inertial and sensor frame, respectively.
The gravity 3D force applied to the 
% structure 
body
is  given by 
\begin{IEEEeqnarray}{RCL}
 \label{eq:g}
 m\bar{g} = m \ls^A R_B^T g ,
\end{IEEEeqnarray}
with $\bar{g},g \in \mathbb{R}^3$ the gravity acceleration expressed w.r.t. the inertial and sensor frame, respectively. The gravity acceleration $\bar{g}$ is constant, so the vector $g$ does not have a constant direction, 
but has a constant norm.

Finally, we make the following main assumption.

\begin{assumption}
 The raw measurements $r$ are taken for static configurations of the 
 rigid body
%  mechanical structure 
 attached to the sensor, i.e. the angular velocity of the frame $S$ is always zero.
 Also, the gravity acceleration $g$ is measured by an accelerometer installed on the rigid body.
%  $\tfrac{d q}{dt} \equiv 0$. 
  Furthermore, no external force-torque, but the gravity force, acts on the rigid body. Hence
\end{assumption}

\begin{IEEEeqnarray}{RCL}
\label{eq:staticWrench}
\rmf &=& 
M(m,c) g, \IEEEyessubnumber  \\
M(m,c) &:=& m
\begin{pmatrix}
1_3 \\
c \times 
\end{pmatrix}. \label{matrixM}  \IEEEyessubnumber 
% = 
% \mathbf{M} \hspace{0.2em}{}^s\mathbf{g}
% =
% \mathbf{M} \hspace{0.2em}
% {}^s\rawval_w {}^w \mathbf{g}
\end{IEEEeqnarray} 

\begin{remark}
We implicitly assume that the accelerometer frame is aligned with the force-torque sensor frame. 
This is a convenient, but non necessary, assumption. 
In fact, if the relative rotation between the sensor frame ${S}$ and the accelerometer frame is unknown, 
it suffices to consider the accelerometer frame as the sensor frame ${S}$.
\end{remark}

Under the above assumptions, what follows proposes a new method for estimating the sensor's offset $o$ 
and for identifying the sensor's calibration matrix $C$ without the need of removing the sensor from the 
hosting system.
