\section{CONCLUSIONS}
\label{conclusions}
In this paper, we addressed the problem of calibrating six-axis force-torque sensors in situ by using the accelerometer measurements. 
The main point was to highlight the geometry behind the gravitational
raw measurements of the sensor, which can be shown to belong to a three-dimensional affine space, and more precisely to a three-dimensional ellipsoid.
Then, we propose a method to identify first the sensor's offset, and then the sensor's calibration matrix. The latter method requires to add sample masses
to the body attached to the sensor, but is independent from the mass and the center of mass of this body. We show that a necessary condition to identify the sensor's calibration matrix is to collect data for more than two 
sample masses. The validation of the method was performed by calibrating the two force-torque sensors embedded in the iCub leg.

The main assumption under the proposed algorithm is that the measurements were taken for static configurations of the rigid body attached to the sensor.
Then, future work consists in weakening this assumption, and developing calibration procedures that hold even for dynamic motions of the rigid body. This
extension requires to use the gyros measurements. In addition, comparisons of the proposed method versus existing calibration 
techniques is currently being investigated, and will be presented in a forthcoming publication.