\section{CONCLUSIONS}
The condition for the \emph{full physical consistency} of the rigid body inertial parameters was introduced, to extend the existing concept of the \emph{physical consistency} of the inertial parameters. A nonlinear parametrization of the inertial parameters that ensures \emph{full physical consistency} was also introduced and a nonlinear optimization on manifolds technique was adapted to solve the resulting nonlinear identification problem. The results have been validated with experiments on the identification of the inertial parameters of the right arm of the iCub humanoid robot.

For the sake of simplicity and for space constraints, in this work, only the case of the identification of a single rigid body was discussed. However, the \emph{full physical consistency} concept and the optimization on manifolds are general concepts, that we plan to integrate with existing techniques for whole body inertial parameters identification in humanoids \cite{ayusawa2014identifiability,jovic2015identification} and for adaptive control of underactuated robots \cite{pucci2015collocated}.