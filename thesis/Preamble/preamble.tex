% ********************Captions and Hyperreferencing / URL **********************

\usepackage[table]{xcolor}
\usepackage[hidelinks]{hyperref}
% \hypersetup{
%    colorlinks=false,
%  linkbordercolor=red,% hyperlink borders will be red
%  pdfborderstyle={/S/U/W 1}% border style will be underline of width 1pt
% }

\RequirePackage[labelsep=space,tableposition=top]{caption}
\renewcommand{\figurename}{Fig.} %to support older versions of captions.sty


% *************************** Graphics and figures *****************************


% FT calibration
\usepackage{subfig}

% Comment large part of text 
\usepackage{comment}

% ********************************** Tables ************************************
\usepackage{booktabs} % For professional looking tables
\usepackage{multirow}



% *********************************** SI Units *********************************
\usepackage{siunitx} 

\usepackage{amsmath}
\usepackage{amsfonts}
\usepackage{amssymb} % e.g., blacksquare
\usepackage{amsthm}
\usepackage{bm}
\usepackage{natbib}

% bsmallmatrix
\usepackage{mathtools}

% IEEEeqnarray
\usepackage{IEEEtrantools}

% overpic
\usepackage{overpic}

\usepackage[utf8]{inputenc}
\usepackage[table]{xcolor}
\usepackage{multirow}
\usepackage{array}
\newcolumntype{L}[1]{>{\raggedright\let\newline\\\arraybackslash\hspace{0pt}}m{#1}}
\newcolumntype{C}[1]{>{\centering\let\newline\\\arraybackslash\hspace{0pt}}m{#1}}
\newcolumntype{R}[1]{>{\raggedleft\let\newline\\\arraybackslash\hspace{0pt}}m{#1}}
\usepackage{hhline}

\usepackage{float}
\usepackage{rotating}

\usepackage[disable]{todonotes}

% Fontawesome
\usepackage{fontawesome}

% setspace
\usepackage{setspace}

% Appendix 
\usepackage[toc,page]{appendix}

% Epigraph 
\usepackage{epigraph}

% *********** Multibody notation ****
\newcommand{\R}{\mathbb{R}}     % Real numbers

\newcommand{\rmv}{{\rm v}}      % 6D velocity
\newcommand{\rmf}{{\rm f}}      % 6D force
\newcommand{\rmh}{{\rm h}}      % 6D momentum
\newcommand{\rma}{{\rm a}}      % 6D acceleration

\newcommand{\rmu}{{\rm u}}      % 6D vector
\newcommand{\rmw}{{\rm w}}      % 6D vector

\newcommand{\rms}{{\rm s}}      % 6D joint motion subspace vector

\DeclareMathOperator{\D}{D}     % differentiation


\newcommand{\bts}{{\bar\times^*}}  % dual cross product 

\newcommand{\SO}{\textrm{SO}}
\newcommand{\so}{\mathfrak{so}}
\newcommand{\SE}{\textrm{SE}}
\newcommand{\se}{\mathfrak{se}}
\newcommand{\ls}{\hspace{0em}}     % For left scripts: use \ls^A_B X^C_D
\newcommand{\phm}{\phantom{-}}   % For alignment 
\newcommand{\ipspace}{\mathfrak{P}}


\DeclareMathOperator{\Ad}{Ad}
\DeclareMathOperator{\ad}{ad}

\newcommand{\bbI}{\mathbb{I}}     % inertia tensor
\newcommand{\bbM}{\mathbb{M}}     % generalized inertia tensor 
                                 
% 
\newcommand{\wrenchTrans}[2]{{\ls_{#1}X^{#2}}}
\newcommand{\twistTrans}[2]{{\ls^{#1}X_{#2}}}

\DeclareMathOperator*{\argmin}{arg\ min.}

\newcommand{\baseParameters}{\pi} 
\newcommand{\genericRegressor}{{Y}}
\newcommand{\genericBigBaseRegressor}{\Upsilon}
\newcommand{\linkSerialization}{{\text{LinkIndex}}}
\newcommand{\baseLink}{{b}}
\newcommand{\nJoints}{n_J}
\newcommand{\nLinks}{n_{L}}
\newcommand{\nDofs}{n}

\newcommand{\baseDepVel}[2]{{#2}^{#1}}
\newcommand{\baseDepTrq}[2]{{#2}_{#1}}
\newcommand{\massMatrix}[1]{ \baseDepTrq{#1}{{M}} }
\newcommand{\coriolisMatrix}[1]{ \baseDepTrq{#1}{{C}} }
\newcommand{\robotVel}[1]{ {\nu}^{#1}}
\newcommand{\robotAcc}[1]{ \baseDepVel{#1}{{\dot{\nu}} } }
\newcommand{\gravityTrq}[1]{ \baseDepTrq{#1}{{G}} }
\newcommand{\linkVel}[1]{{\rmv}_{#1}}
\newcommand{\linkAcc}[1]{{\dot{\rmv}}_{#1}}
\newcommand{\jointPos}{s}
\newcommand{\jointTrq}{\tau}
\newcommand{\jointVel}{{\dot{s}}}
\newcommand{\jointAcc}{{\ddot{s}}}
\newcommand{\jacobian}[2]{\baseDep{#1}{{J}}_{#2}}
\newcommand{\baseJacobian}[2]{\jacobian{#1}{#2}^B}
\newcommand{\jointJacobian}[2]{\jacobian{#1}{#2}^L}
\newcommand{\newBaseFrame}{D}

\newcommand{\bodyVel}{V}
\newcommand{\bodyLinkVel}[1]{\bodyVel_{#1}}
\newcommand{\base}{b}
\newcommand{\basePos}{\homTrans{\inertialFrame}{\bodyFrame}}
\newcommand{\invBasePos}{\homTrans{\bodyFrame}{\inertialFrame}}
\newcommand{\linkSet}{\mathfrak{L}}
\newcommand{\jointSet}{\mathfrak{J}}
\newcommand{\baseVel}{\bodyVel_{\bodyFrame}}
\newcommand{\baseAcc}{\dot{\baseVel}}
\newcommand{\extWrench}[1]{f_{#1}^{ext}}
\newcommand{\robotPos}{q}
\newcommand{\inertialFrame}{A}
\newcommand{\bodyFrame}{B}
\newcommand{\bodySpatialInertia}{\Lambda}
\newcommand{\bodyPos}{H}
\newcommand{\ffVel}{\nu} 
\newcommand{\ffTrq}{\phi}
% Todo find 
\newcommand{\netFT}{\phi}

\newcommand{\homTrans}[2]{{}^{#1}H_{#2}}

\newcommand{\offset}{b}

% ********************************** Theorems ************************************

% Assumption is thing that is accepted as true or as certain to happen, without proof.
\newtheorem{assumption}{Assumption}
\numberwithin{assumption}{chapter}

% Remark is a written and isolated comment to a part of the text 
\newtheorem{remark}{Remark}
\numberwithin{remark}{chapter}

% Definition is a statement that introduce a new concept/quantity
\newtheorem{definition}{Definition}
\numberwithin{definition}{chapter}


% Then we have a hierarchy of statement that need a demonstration:
% Theorem <-- Proposition <-- Lemma <-- Corollary 
% A Corollary is always attached to a Theorem, and the 
% demonstration of a Corollary can be omitted because it
% it supposed to be a straightforward  consequence of a result of 
% a theorem 
\newtheorem{theorem}{Theorem}
\numberwithin{theorem}{chapter}

\newtheorem{proposition}{Proposition}
\numberwithin{proposition}{chapter}

\newtheorem{lemma}{Lemma}
\numberwithin{lemma}{chapter}

\newtheorem{corollary}{Corollary}
\numberwithin{corollary}{chapter}

% Property is a "natural" consequence of some assumption or definition 
\newtheorem{property}{Property}
\numberwithin{property}{chapter}

% **************** FT Calibration stuff
\newcommand{\nrofsg}{6}
\newcommand{\calibmat}{{C}}
\newcommand{\shapemat}{{S}}
\newcommand{\rawval}{{r}}
\newcommand{\senswrench}{{}^s{w}}
\newcommand{\sensgrav}{{}^s{g}}
\newcommand{\offsetwrench}{{}^{o_w}}
\newcommand{\offsetraw}{{o_r}}

% **************** Change base stuff
\newcommand{\velBaseTrans}[2]{\ls^{#1}T_{#2}}
\newcommand{\baseTransform}[2]{\velBaseTrans{#1}{#2}}
\newcommand{\forceBaseTrans}[2]{\ls_{#1}T^{#2}}
\newcommand{\baseTransformRec}[2]{\forceBaseTrans{#1}{#2}}

% **************** Weird accent stuff
\newcommand{\Poincare}{Poincar\'e\ }

% *****************************************************************************
% *************************** Bibliography  and References ********************

% ******************************************************************************
% ************************* User Defined Commands ******************************
% ******************************************************************************

% *********** To change the name of Table of Contents / LOF and LOT ************

%\renewcommand{\contentsname}{My Table of Contents}
%\renewcommand{\listfigurename}{My List of Figures}
%\renewcommand{\listtablename}{My List of Tables}


% ********************** TOC depth and numbering depth *************************

\setcounter{secnumdepth}{2}
\setcounter{tocdepth}{2}


% Dedicatory 
% \newcommand\bhrule{\typeout{------------------------------------------------------------------------------}}

% \newcommand\btypeout[1]{\bhrule\typeout{\space #1}\bhrule}


\newcommand\dedicatory[1]{
% \btypeout{Dedicatory}
\thispagestyle{plain}
% \null\vfil
 %\p@
\begin{center}{\Large \sl #1}\end{center}
\vskip 2em
% \vfil\null
% \cleardoublepage
}
